\clearpage
\section{ソースコードから起動する方法}

\subsection{Windows}

\begin{enumerate}
\item \url{https://github.com/takumak/tuna/archive/master.zip}
  をダウンロードして展開する.
  もしくは\code{git clone https://github.com/takumak/tuna.git}してもいい.
\item \url{https://www.python.org/downloads/windows/}
  からPythonをダウンロードしてインストールする.
  Python 3.6.5rc1やPython 3.7.0b2など, バージョン番号の後ろにrcやa, bがついているものはテスト版なので,
  Python 3.6.4など, バージョン番号のみのものを選ぶ.
  exeファイルを作りたい場合, pyinstallerの相性問題によりPython 3.5系の最新版を推奨する.
  x86とx86-64などは何でもいいが, Windows x86-64 executable installerを推奨する.
\item \url{https://www.lfd.uci.edu/~gohlke/pythonlibs/}
  からnumpy, scipy, PyOpenGL, PyOpenGL\_accelerateをダウンロードして
  \code{tuna{\textbackslash}build{\textbackslash}win} に入れる.
  Python 3.5系をインストールした場合はcp35, 3.6系ならcp36がファイル名に含まれているものを選ぶ.
  x86を選択した場合はwin32, x86-64を選択した場合はwin\_amd64がファイル名に含まれているものを選ぶ.
  どのファイルをダウンロードすべきかわからない場合, 後で \code{python build.py}
  を実行したらダウンロードすべきファイルを指示されるので, それをダウンロードする.
\item コマンドプロンプトを開いて以下のコマンドを実行する.
\end{enumerate}

\begin{codeb}{batch}
  cd tuna-master|\textbackslash|build|\textbackslash|win
  python build.py
  cd ..|\textbackslash|..
  .|\textbackslash|build|\textbackslash|win|\textbackslash|venv|\textbackslash|Scripts|\textbackslash|activate.bat
  python src|\textbackslash|tuna.py
\end{codeb}

\subsection{Mac/Linux}

ターミナルで以下のコマンドを実行する.
事前に\href{https://github.com/pyenv/pyenv}{pyenv}を導入していれば,
\code{tools/make\_virtualenv.bash} が自動的にテスト済みのバージョンを導入する%
(\code{pyenv install 3.5.5;pyenv local 3.5.5}).

\begin{codeb}{bash}
  $ git clone https://github.com/takumak/tuna.git
  $ cd tuna
  $ ./tools/make_virtualenv.bash
  $ ./tools/run_in_virtualenv.bash
\end{codeb}
